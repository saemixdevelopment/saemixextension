\chapter{Introduction} \label{chapter_introduction}

\monolix is a package for {\sf R} to perform parameter estimation in non-linear mixed effect models. It has been hosted on the CRAN since version 0.95 in June 2011.

\section{The objectives}

The objectives of \monolix~are to perform:
\begin{enumerate}
\item parameter estimation for non-linear mixed effects models
\begin{itemize}
\item[-] computing the maximum likelihood estimator of the population parameters, without any approximation of the model (linearization, quadrature approximation, \ldots), using the Stochastic Approximation Expectation Maximization (SAEM) algorithm,
\item[-] computing standard errors for the maximum likelihood estimator
\item[-] computing the conditional modes, the conditional means and the conditional standard deviations of the individual parameters, using the Hastings-Metropolis algorithm
\end{itemize}
\item goodness of fit plots
\item model selection
\begin{itemize}
\item[-] comparing several models using some information criteria (AIC, BIC)
\item[-] testing hypotheses using the Likelihood Ratio Test
\item[-] testing parameters using the Wald Test
\end{itemize}
\end{enumerate}
The R package \monolix~is an implementation of the Stochastic Approximation Expectation Maximization (SAEM) algorithm in \R~\cite{R}, developed by K\"uhn and Lavielle~\cite{Kuhn05}, and implemented in the {\sc Monolix} software available in Matlab and as a standalone software for Windows and Linux~\cite{LavielleMonolix}.

The current version of the R version of \monolix~handles only analytical functions. The following features have not yet been implemented in the R package \monolix, but are available in the {\sc Monolix} software:
\begin{itemize}
\item categorical covariates with more than 2 categories
\item models defined with differential equations
\item multi-response model
\item left censored data
\item interoccasion variability
\item prior distribution for the random effects
\item complex variables, including discrete data or repeated time to events
\item hidden Markov models
\item mixture models
\item autocorrelation of the residuals
\end{itemize}

Theoretical analysis of the algorithms used in this software can be found in \cite{Delyon, samson_jspi06, Kuhn01, Kuhn05}. Several application of SAEM in agronomy \cite{Makowski06}, animal breeding \cite{Jaffrezic06} and PKPD analysis \cite{Comets07, Lavielle07, samson_csda06, samson_sim06a, Bertrand09} have been published by several members of the {\sf Monolix} group. Several applications to PKPD analysis were also proposed during the last PAGE (Population Approach Group in Europe) meetings (\cite{page06b, page05a, page03, page04a, page06c, page05b} as well as a comparison of estimation algorithms \cite{page05c}, ({http://www.page-meeting.org}).

The present document describes the non-linear mixed effects models (section~\ref{sec:models}) and the algorithms used in this package (section~\ref{sec:methods}). The library's inputs and outputs are described in section~\ref{sec:package}. Section~\ref{sec:examples} shows some examples made available in the library (section).

\section{Installation and legalese} \label{sec:installation}

\subsection{Installation}

\subsection{Citing \monolix}

\hskip 18pt If you use this program in a scientific publication, we would like
you to cite the following reference:
\begin{quotation}
\noindent 
Comets E, Lavenu A, Lavielle M (2017). SAEMIX, an R version of the SAEM algorithm. Journal of Statistical Software, 80:1-41.
\end{quotation}

A BibTeX entry for \LaTeX$\;$ users is:

\begin{verbatim}
@Article{,
author	={Emmanuelle Comets and Audrey Lavenu and Marc Lavielle},
title	={Parameter estimation in nonlinear mixed effect models using saemix, an {R} implementation of the {SAEM} algorithm},
volume	={80},
pages	={1--41},
journal	={Journal of Statistical Software},
year	=2017	}
\end{verbatim}

